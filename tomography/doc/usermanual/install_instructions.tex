\documentclass[a4paper,10pt]{scrartcl}
\usepackage[utf8x]{inputenc}

%opening
\title{Installation instructions for MuhRec2 on Linux}
\author{A. Kaestner}

\begin{document}

\maketitle

\section{System requirements}
To install MuhRec2 on your system it is recommended that it is running with
Ubuntu 11.04 or more, using the bash-shell. This is the easiest\ldots
The current distribution is compiled and tested for a 64-bit OS, i.e. don't try
to use it with a 32-bit system!

\section{Installation}
The installation of the application is done with the following steps:
\begin{enumerate}
 \item Extract the muhrec directory in the distribution file to the \verb+/opt+
directory. \\
\verb+cd /; sudo tar -xvf muhrec2_YYMMDD.tar+
\item Set environment variables.
\begin{description}
 \item[Temporary] Call \verb+source /opt/muhrec/install/set_enviroment.sh+
 \item[Single user] Add the previous source call to your \verb+.bashrc+-file
\item[All users] Add the source call to the \verb+/etc/bash.bashrc+
\end{description}
\item Start the application \verb+muhrec2+ to work with the GUI or
\verb+muhrec2 -f parameters.xml+ for command-line operation. The parameter file
is the same type as saved from the GUI.
\end{enumerate}

\section{Good luck}
This software is distributed with no guarantee of accuracy or functionality.
If the software would show a malfunction please report it to me, then I will see
what I can do about it. I am also open for suggestions for improvements. There
is no guarantee that it will be implemented, that is a matter of time\ldots 


\end{document}
